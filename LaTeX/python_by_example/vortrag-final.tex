\documentclass[10pt]{beamer}

\usepackage[ngerman]{babel}
\usepackage[T1]{fontenc}
\usepackage{ucs}            % Bis Edgy benötigt, später nicht mehr
\usepackage[utf8]{inputenc}

%\usepackage{beamerthemesplit}
\usepackage{graphicx}
\usepackage{xcolor}
%%%%%%%%
% listings
\usepackage{listings}
\usepackage{pylisting}

%%%%%%%%
% For beamer class
\mode<presentation>
{
% Moegliche Themes
%\usetheme{
%       AnnArbor | Antibes | Bergen |
%       Berkeley | Berlin | Boadilla |
%       boxes | CambridgeUS | Copenhagen |
%       Darmstadt | default | Dresden |
%       Frankfurt | Goettingen |Hannover |
%       Ilmenau | JuanLesPins | Luebeck |
%       Madrid | Malmoe | Marburg |
%       Montpellier | PaloAlto | Pittsburgh |
%       Rochester | Singapore | Szeged |
%       Warsaw
%}
% evtl AnnArbor, Goettingen, Madrid, Marburg, Hannover, CambridgeUS, boxes
  \usetheme{AnnArbor}

% Moegliche Farben
%\usecolortheme{
%       albatross | beaver | beetle |
%       crane | default | dolphin |
%       dove | fly | lily | orchid |
%       rose |seagull | seahorse |
%       sidebartab | structure |
%       whale | wolverine
%}
 % evtl beaver, rose, 
\usecolortheme{sidebartab}

  \setbeamercovered{transparent}
}


\title{Python by Example}
\date{}
\author{Thorsten Hillebrand}
\date{09. 06. 2008}
\institute{HAW Hamburg}

\begin{document}

\begin{frame}
\titlepage

 \begin{center}
%  \includegraphics[width=180,bb=0 0 608 206]{python-logo.png}
  \includegraphics[width=80mm]{python-logo.png}
  % python-logo.png: 601x203 pixel, 71dpi, 21.46x7.25 cm, bb=0 0 608 206
 \end{center}

\end{frame}

%\frame{\tableofcontents}
\begin{frame}
 \tableofcontents[circle]
\end{frame}


\section{Motivation}
 \begin{frame}
  \frametitle{Motivation}
  \begin{itemize}
%   \item Wie bin ich zu Python gekommen?
   \item Wer setzt Python ein?
    \begin{itemize}
     \item Als Scriptsprache
      \begin{itemize}
       \item OpenOffice.org
       \item Blender
       \item Maya
       \item Gimp
      \end{itemize}
     \item realisierte Softwareprojekte
      \begin{itemize}
       \item Zope Application Server
       \item NASA
       \item Google setzt Python umfassend ein
       \item YouTube ist nahezu vollständig in Python geschrieben
       \item Industrial Light and Magic
       \item offizieller BitTorrent-Client
       \item CCP Games
       \item EVE-Online
      \end{itemize}
    \end{itemize}
   \item Editoren und IDEs
    \begin{itemize}
     \item PyDev http://wiki.python.org/moin/PyDev
     \item eric http://www.die-offenbachs.de/eric/index.html
     \item Boa Construktor http://boa-constructor.sourceforge.net
    \end{itemize}
  \end{itemize}

 \end{frame}


%\section{Beispiel}
% \begin{frame}[fragile]
%  \frametitle{Beispiel}
%   \begin{lstlisting}
%values = ([number for number in range(1,1000) if number % 5 == 0 \
%            or number % 3 == 0])
%print reduce((lambda x,y: x+y), values)%
%
%   
%def fib(n):
%   if n == 0 or n == 1:
%      return n
%   else:
%      return fib(n-1) + fib(n-2)%
%
%print reduce(lambda x,y: x*y, range(1,6))
%
%a, b = 0, 1
%while b < 10:
%    print(b)
%    a, b = b, a+b
%   \end{lstlisting}
%\end{frame}


\section{Überblick}
 \begin{frame}[fragile]
  \frametitle{Überblick}
   \begin{itemize}
    \item Erscheinungsjahr 1990
    \item Guido van Rossum (Python Software Foundation)
    \item Aktuelle Version: 2.5.2 (22. Februar 2008)
    \item Typisierung: stark, dynamisch (Duck Typing)
    \item Plattformunabhängig (Bytecode)
    \item Python License
    \item Garbage Collection
    \item Alles ist ein Objekt
   \end{itemize}
   \begin{lstlisting}
values = ([number for number in range(1,1000) if number % 5 == 0 \
            or number % 3 == 0])
print reduce((lambda x,y: x+y), values)

def fib(n):
   if n == 0 or n == 1:
      return n
   else:
      return fib(n-1) + fib(n-2)

print reduce(lambda x,y: x*y, range(1,6))
   \end{lstlisting}
\end{frame}

\subsection{Besondere Merkmale}
 \begin{frame}[fragile]
  \frametitle{Besondere Merkmale}
   \begin{itemize}
    \item Sehr kompakte Programmtexte
    \item Einrückung kennzeichnet Blöcke
    \item Objektorientiert, imperativ, funktional
    \item Keine Deklaration
    \item Minimalistisch
    \item Ganze Zahlen beliebiger Länge
   \end{itemize}
   \begin{lstlisting}
x = y = z = 5         # 5 5 5
x, y, z = 1, 2, 3     # 1 2 3
print x < y < z       # True

a, b = 23, 42         # 23 42
a, b = b, a           # swap
print a, b            # 42 23
print "The answer is %d" % a  # The answer is 42

print reduce(lambda x,y: x*y, range(1,99))
#942689044888324774562618574305724247380969376407895166349423
#877729470707002322379888297615920772911982360585058860846042
#9412647567360000000000000000000000   
   \end{lstlisting}
\end{frame}

\subsection{Python vs. Java}
\begin{frame}
  \frametitle{Python vs. Java}
  \begin{itemize}
   \item import - import
   \item self - this
   \item None - null
   \item dynamisch - statisch
   \item kompakt - gesprächig
   \item Einrückung kennzeichnet Blöcke - Klammern kennenzeichnen Blöcke
  \end{itemize}
\end{frame}

\section{Sequenzen}
\begin{frame}[fragile]
  \frametitle{Sequenzen}
  \begin{itemize}
   \item Folge mehrerer Objekte
   \item Listen sind veränderbar
   \item Strings und Tupel sind nicht änderbar
   \begin{itemize}
    \item Keine Zuweisung s[i] = \dots
    \item Kein Anhängen und Löschen von Elementen
    \item Funktionen wie upper liefern einen neuen String zurück
   \end{itemize}
   \item Slicing
  \end{itemize}
\begin{lstlisting}
a = [1, "spam", 9.0, 42]        # [1, 'spam', 9.0, 42]
print 42 in a                   # True
print "spam" not in a           # False
print a[1]                      # spam
print a[0:2]                    # [1, 'spam']
print a[2:]                     # [9.0, 42]
print a[:]                      # [1, 'spam', 9.0, 42]
print a[-1]                     # 42
print min(a), max(a), len(a)    # 1 spam 4

t = (2,4,6)
print t                         # (2, 4, 6)
x, y, z = t
print x, y, z                   # 2 4 6
\end{lstlisting}
\end{frame}


\subsection{Strings}
 \begin{frame}[fragile]
  \frametitle{Strings}
   \begin{itemize}
    \item kurze Zeichenketten  \texttt{`` oder '}
    \item lange Zeichenketten  \texttt{'' '' ''  oder ' ' '}
    \item Unicode \texttt{u' oder u"}
    \item Raw \texttt{r`` oder r'}
   \end{itemize}

   \begin{lstlisting}
> print 'NI'*3
NININI
> print 'sp\nam'
sp
am
> print r'sp\nam'
sp\nam
> s = """Hello 
. world"""
> print s
Hello
world
   \end{lstlisting}
\end{frame}

\subsection{Listen}
\begin{frame}[fragile]
  \frametitle{Listen}
  \begin{itemize}
   \item Items beliebiger Datentypen
   \item List als Stack benutzen
\begin{lstlisting}
stack = [3,4,5]
stack.append(6)           # [3, 4, 5, 6]
print stack.pop()         # 6
print stack               # [3, 4, 5]
\end{lstlisting}
   \item List als Queue benutzen
    \begin{lstlisting}
queue = ["Eric", "John"]
queue.append("Terry")     # ['Eric', 'John', 'Terry']
print queue.pop(0)        # Eric
print queue               # ['John', 'Terry']
\end{lstlisting}
    \item List Comprehension
\begin{lstlisting}
> [str(round(355./113, i)) for i in range(1, 7)]
['3.1', '3.14', '3.142', '3.1416', '3.14159', '3.141593']
> [i for i in range(50) if i % 7 == 0]
[0, 7, 14, 21, 28, 35, 42, 49]
  \end{lstlisting}
   \end{itemize}
\end{frame}

\section{Dictionaries}
\begin{frame}[fragile]
  \frametitle{Dict}
  \begin{itemize}
   \item Folge von Wertepaaren
   \item Schneller Zugriff über keys
  \end{itemize}
  \begin{lstlisting}
knights = {'gallahad' : 'the pure', 'robin' : 'the brave'}

print 'gallahad' in knights      # True
print 'lancelot' not in knights  # True

knights['gallahad'] = 'The Pure'
print knights['gallahad'] # The Pure
print knights.keys()      # ['gallahad', 'robin']
print knights.values()    # ['The Pure', 'the brave']
print knights.items()     # [('gallahad', 'The Pure'), ('robin', 'the brave')]

knights2 = {'lancelot' : 'not-quite-so-brave'}
knights.update(knights2)

print 'lancelot' in knights    # True

for k, v in knights.items():
    print k, v
#gallahad The Pure
#robin the brave
#lancelot not-quite-so-brave...
  \end{lstlisting} 
\end{frame}



\section{Kontrollstrukturen}
\begin{frame}
 \tableofcontents[current]
\end{frame}

\begin{frame}[fragile]
  \frametitle{Kontrollstrukturen}
  \begin{lstlisting}
zahl = 42
counter = 0
while True:
    geraten = int(raw_input('Zahl: '))
    counter += 1
    if zahl == geraten:
        print 'Richtig nach %d Versuchen.' % counter
        break
    elif geraten < zahl:
        print 'Die Zahl ist etwas hoeher.'
        continue
    else:
        print 'Die Zahl ist etwas niedriger.'
        continue
    print 'Diese Stelle wird nie erreicht'
else:
    print "Schleife vollstaendig durchlaufen"
print "Programmende."
  \end{lstlisting}
  \begin{lstlisting}[style=Shell]
Zahl: 1
Die Zahl ist etwas hoeher.
Zahl: 50
Die Zahl ist etwas niedriger.
Zahl: 42
Richtig nach 3 Versuchen.
Programmende.  
  \end{lstlisting}
\end{frame}

\section{Schleifen mit Bedingung}
 \begin{frame}[fragile]
  \frametitle{Schleifen mit Bedingungen}
   \begin{itemize}
    \item break
    \item continue
    \item else
     \begin{lstlisting}
x = 2
while x <= 1000000:
    x = x*2
else:
    print "Kleinste Zweierpotenz, "
    print "die groesser als 1000000 ist: ", x
#Kleinste Zweierpotenz,
#die groesser als 1000000 ist:  1048576  
     \end{lstlisting}

   \end{itemize}
\end{frame}

\section{Iteration}
\begin{frame}[fragile]
  \frametitle{Iteration}

   \begin{itemize}
     \item \texttt{range([start,] stop [,step])}
      \begin{lstlisting}
 print range(6)               # [0, 1, 2, 3, 4, 5]
 print range(0, 10, 2)        # [0, 2, 4, 6, 8]
 print range(-10, 10, 2)      # [-10, -8, -6, -4, -2, 0, 2, 4, 6, 8]
       \end{lstlisting}
     \item for-each Schleife
      \begin{lstlisting}
lst = [6,3,9,15,2,5,27,0] + range(13,50,7)
numbers = lst[:]
numbers.sort()
for item in reversed(numbers):
    print item,
# 48 41 34 27 27 20 15 13 9 6 5 3 2 0  
      \end{lstlisting}
     \item Dateien
      \begin{lstlisting}
for line in open('file.txt'):
    print line,
#Python
# Batteries included  
      \end{lstlisting}
   \end{itemize}
\end{frame}


\section{Ausnahmen}
 \begin{frame}[fragile]
  \frametitle{Ausnahmen}
   \begin{lstlisting}
while True:
    try:
        num = int(raw_input("Zahl: "))
    except ValueError:
        print "Das war keine Zahl"
    except:
        print "Anderer Fehler"
    else:
        print "Eingegebene Zahl: ", num
        break
    finally:
        print "wichtiger Code wird ausgefuehrt"

print num
   \end{lstlisting}

   \begin{lstlisting}[style=SHELL]
Zahl: abc
Das war keine Zahl
wichtiger Code wird ausgefuehrt
Zahl: x
Das war keine Zahl
wichtiger Code wird ausgefuehrt
Zahl: 42
Eingegebene Zahl:  42
wichtiger Code wird ausgefuehrt
42   
   \end{lstlisting}

\end{frame}

\section{Funktionen}
\begin{frame}[fragile]
 \frametitle{Funktionen}
 \begin{lstlisting}
def doit(x=10, y=10, z=10):
    sum = x + y + z
    return sum, x, y, z

t = doit()
print t                     # (30, 10, 10, 10)
print type(t)               # <type 'tuple'>
print doit(1, 2, 3)         # (6, 1, 2, 3)
print doit(z=3, y=2, x=1)   # (6, 1, 2, 3)
sum, x, y, z = doit(1, z=3)
print sum                   # 14
\end{lstlisting}
\begin{lstlisting}
def keys(*args, **kwargs):
    for i in args:
        print i
    for k,v in kwargs.items():
        print k,v

keys(1, 2, "test", x=20, y=30, python="holy grail")
 \end{lstlisting}
\begin{lstlisting}[style=SHELL]
1
2
test
y 30
x 20
python holy grail
\end{lstlisting}

\end{frame}

\section{Batteries included}
 \begin{frame}[fragile]
  \frametitle{Batteries included}
   \begin{itemize}
    \item Große Standardbibliothek
    \item Funktionale Programmierung
    \begin{itemize}


     \item \texttt{map(function, iterable ...)}
      \begin{lstlisting}
import math
a = [9, 16, 25, 36]
print map(math.sqrt, a)                # [3.0, 4.0, 5.0, 6.0]
print map(lambda x: x*2, a)            # [18, 32, 50, 72]
print map(math.pow, a, [2 for i in a]) # [81.0, 256.0, 625.0, 1296.0]
     \end{lstlisting}
    \item \texttt{filter(function, iterable)}
     \begin{lstlisting}
a = range(10)
print filter(lambda x: x % 2, a)        # [1, 3, 5, 7, 9]
     \end{lstlisting}
    \item \texttt{reduce(function, iterable [,init])}
     \begin{lstlisting}
def add(x,y):
    return x+y
a = range(10)
print reduce(add, a)                    # 45
print reduce(lambda x,y: x+y, a)        # 45
     \end{lstlisting}
    \end{itemize}
   \end{itemize}
\end{frame}

%\section{Ein- und Ausgabe}
%\begin{frame}[fragile]
%  \frametitle{Ein- und Ausgabe}
%  \begin{lstlisting}
%   todo Dateien, console
%  \end{lstlisting}
%\end{frame}

\section{OO}
\begin{frame}[fragile]
  \frametitle{OO}
  \begin{lstlisting}
class Point(object):
    def __init__(self, x, y):
        self.x = x
        self.y = y

    def say_hello(self):
        print "Hello World"

    def __add__(self, other):
        return Point(self.x+other.x, self.y+other.y)

    def __eq__(self, other):
        return self.x == other.x and self.y == other.y

    def __str__(self):
        return "%d %d" % (self.x, self.y)

if __name__ == "__main__":
    p1 = Point(5, 10)
    p2 = Point(5, 10)
    p1.say_hello()         # Hello World
    print p1 == p2         # True
    p2 = Point(100, 200)
    print p1 == p2         # False
    print p1 + p2          # 105 210
  \end{lstlisting}
\end{frame}

\subsection{OO - Vererbung}
\begin{frame}[fragile]
  \frametitle{OO - Vererbung}
  \begin{lstlisting}
import threading, time
class Counter(threading.Thread):
    def __init__(self):
        threading.Thread.__init__(self)
        self.counter = 0

    def run(self):
        while self.counter < 10:
            self.counter += 1
            time.sleep(1)
            print self.counter

c = Counter()
c.start()
print "Thread gestartet"
c.join()
print "Thread Ende"
  \end{lstlisting}
\begin{lstlisting}[style=SHELL]
Thread gestartet
1
.
9
10
Thread Ende 
\end{lstlisting}

\end{frame}

\section{Fazit}
 \begin{frame}
  \frametitle{Fazit}
   \begin{itemize}
    \item leicht zu lernen
    \item minimale Syntax
    \item große Standardbibliothek
    \item gute Verbreitung
    \item Gargabe Collection
    \item Open Source
    \item Einrückung sichert die Wartbarkeit
    \item Funktionen mit Defaultwerte
   \end{itemize}
\end{frame}

 \begin{frame}
  \frametitle{Literatur und Links}
   \begin{itemize}
    \item Literatur
     \begin{itemize}
      \item Python gepackt, {\scriptsize mitp Verlag, ISBN: 3-8266-1512-3}
      \item Programming Python, {\scriptsize O'Reilly Verlag, ISBN 0-596-00925-9}
      \item Dive Into Python
       \begin{itemize}
        \item kostenlos unter http://www.diveintopython.org
       \end{itemize}
      \item Think Python
       \begin{itemize}
        \item kostenlos unter http://www.greenteapress.com/thinkpython
       \end{itemize}
     \end{itemize}
    \item Links
     \begin{itemize}
      \item http://docs.python.org
      \item http://www.python.org
      \item http://de.wikipedia.org/wiki/Python\_(Programmiersprache)
      \item http://www.python-forum.de
      \item http://projecteuler.net
      \item http://www.pythonchallenge.com
     \end{itemize}
   \end{itemize}
\end{frame}

\begin{frame}
 \frametitle{Ende}
  \begin{center}
\begin{large}Vielen Dank für die Aufmerksamkeit!\end{large}
  \end{center}

  \begin{center}
   \includegraphics[width=80mm]{python-logo.png}
   % python-logo.png: 601x203 pixel, 71dpi, 21.46x7.25 cm, bb=0 0 608 206
  \end{center}
\end{frame}


\end{document}
